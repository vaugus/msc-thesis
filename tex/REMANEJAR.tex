## introdução - conclusão

\section*{Publications}

\begin{itemize}
    \item Victor Augusto Alves Catanante, Odemir Martinez Bruno, and João do Espírito Santo Batista Neto. ``Laplacian of Gaussian-based Multi-focus Bright-field Microscopy Image Fusion''. \textbf{Submitted to The 27th International Conference on Systems, Signals and Image Processing - IWSSIP 2020}.
    
    \item \cite{catanante2020frequency} Victor Augusto Alves Catanante, Odemir Martinez Bruno, and João do Espírito Santo Batista Neto. ``Frequency domain kurtosis-based no-reference image quality assessment for bright-field microscopy images''. \textbf{To be submitted}.
\end{itemize}

## introdução - metodologia

When applied to our proposed image datasets (which comprise both totally out of focus and partially sharp images), it will provide an estimate of the quality of images concerning sharpness and consequently allow the selection of suitable images to the subsequent fusion process;

In this work, the \emph{stereomicroscopy}, \emph{bright-field microscopy} and the \emph{z-stacking} techniques are used in order to capture images.

## cap 2 - cap 3

A single lens or a set of lenses (most of the optical systems are more complicated than a single lens) have the \emph{depth of field} and the \emph{depth of focus} properties. Although the terms appear to be similar, the former relates to objects and the latter, to images \cite{davidson2002optical}. For every system, there is a focal plane in which the formed images will be sharp. Depth of Field is the tolerance for the object focal plane that may produce sharp images, while Depth of Focus dictates the same tolerance for the image focal plane. In other words, depth of field is the zone in the real world that would yield an acceptably sharp image and depth of focus is the same idea for the imaging sensors or for plotting the image.

## remover referencias

NIXON AGUADO 2019
NAYAR BEN-Ezra 2004

apendice - materiais e metodos

