This work proposed a no-reference DFT-based image quality metric that produced reliable results, which highly correlate with the labels obtained by subjective analysis. Additionally, a multi-focus image fusion algorithm for bright-field microscopy z-stacks that explores the energy of edges extracted with a Laplacian of Gaussian operator proved to be an efficient way to perform image fusion. Both of our hypotheses that methods based on frequency domain information and the Laplacian of Gaussian operator were confirmed experimentally in the performance comparison with related methods.

Concerning our IQA method, the implementation is efficient in terms of computational performance and it may be extended to other applications such as auto-focus systems. Future work on this method comprises several possibilities. The improvement of the computational performance of the method by refactoring the implementation and applying parallel programming techniques and the integration of the software with hardware devices may be the basis for new imaging solutions. The study and empirical evaluation of other methods to estimate the set of eligible images rather than $z$-score, which includes more advanced statistical methods and machine learning methods, may provide even better results.

The most obvious improvement is to implement it in C++ in order to improve computational performance since our Python implementation is a prototype. Next, we should study and evaluate the availability of information in each channel of the RGB images, as it may suppress the need of a grayscale conversion and also may yield better results. Additionally, different smoothing functions and other detection algorithms should be evaluated, and those should also be robust to noise and enhance the difference between blurred and sharp regions. Finally, the research on the availability of different fusion rules rather than the energy of edges and the use of frequency-domain convolution may lead to better results and faster computation, respectively.

The set of mathematical methods and techniques applied in this work are standards in image processing. All of them have been extensively explored during the past few decades, and this work is another example which underpins the strength of such methods. However, there is still a lot to be explored in terms of tuning parameters and extracting more from the theoretical side of each technique. This approach is important, since it helps to find limitations and also to build novel tools.

An interesting and important insight from this work is the use of a full-reference image quality metric in order to evaluate a no-reference one. This evolves to a brand new area of study, since there are many benchmark methods, e.g. the MSSIM itself, which can be applied in this scenario. The evaluation of a blind metric may be more effective under a combination of several non-blind metrics. The whole IQA area would benefit from a novel validation protocol based on well-known methods.


\section*{Publications}

\begin{itemize}
    \item \cite{catanante2020frequency} Victor Augusto Alves Catanante, Odemir Martinez Bruno, and João do Espírito Santo Batista Neto. ``Frequency domain kurtosis-based no-reference image quality assessment for bright-field microscopy images''. \textbf{To be submitted}.
\end{itemize}
