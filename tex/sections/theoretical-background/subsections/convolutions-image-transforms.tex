As seen in Chapters \ref{chapter:fundamentals-of-optics-and-light-microscopy} and \ref{chapter:blur-characterization-and-image-formation}, the processes of image formation and acquisition concerning linear systems are subject to some operations, i.e. the convolution, the \sigla{FT}{Fourier Transform} and its continuous and discrete versions, that modify the original representation of the scene. The linear system theory provides mathematical tools to explore those operations and others, such as sampling, filtering and enhancement; it describes the behavior of electrical circuits and optical systems \cite{castleman1996digital}.

According to \citeonline{bracewell2000fourier}, the convolution of two arbitrary functions $s$ and $t$ results in another function $r$ is, with notation adjustments, defined by the integral

\begin{equation}
\label{eqn:one_dimensional_convolution}
r(x) = \int_{-\infty}^{\infty}s(u) t(x - u) du,
\end{equation}

\noindent where $x$ is the one-dimensional coordinate and $u$ is a shifting parameter. In other words, this procedure is compared to moving a 180 degrees-rotated filter mask over the function and computing the sum of products at each location \cite{gonzalez2018digital}. Hence, the shifting parameter represents the slide of the filter mask along the function. As seen in Chapter~\ref{chapter:blur-characterization-and-image-formation}, convolution is responsible for the image formation and acquisition processes, but it also covers several other applications, e.g. smoothing, sharpening and reducing noise in images. Similarly to the one-dimensional case and pursuant to \citeonline{castleman1996digital}, the two-dimensional convolution is denoted by

\begin{equation}
\label{eqn:two_dimensional_convolution}
r(x,y) = \int_{-\infty}^{\infty}
         \int_{-\infty}^{\infty}
         s(u,v) t(x - u, y - v) du dv,
\end{equation}

\noindent where $u$ and $v$ are shifting parameters, $\ast$ denotes the convolution. The \emph{spatial domain} is the subset of the real plane where functions like $r$, $s$ and $t$ are spanned, and $(x,y)$ as points within such subset are named spatial variables; consequently, any mathematical operation that employs pixels from this subset are named spatial domain techniques. Concerning the digital image processing applications, which deals with images as matrices of pixels, the discrete two-dimensional convolution for an image $f(x,y)$ and a function $h(x,y)$ is given by 

\begin{equation}
\label{eqn:2d_discrete_convolution}
g(x,y) = h(x,y) \ast f(x,y) = 
        \sum_{m=-a}^{a}
        \sum_{n=-b}^{b}
        f(m,n) h(x-m,y-n)
\end{equation}

\noindent where $a = (m-1)/2$ and $b = (n-1)/2$, given that the function $h(x,y)$ is considered to be a two-dimensional filter of size $m \times n$ \cite{gonzalez2018digital}.

Convolution, together with several other operations employed in this work, operate directly on the spatial domain by modifying pixel values based on mathematical constraints. Some of those operations may have issues that hinder the operation, e.g. the computation time of a convolution process should be finite, otherwise its use is impractical; this is one of the reasons why \emph{image transforms} are widely used. They encompass any group of mathematical operations that transfers the input signal or image from its domain to the transform domain \cite{gonzalez2008digital}. Let $s$ be an arbitrary two-variable function, $t_{f}$ and $t_{i}$ be the forward and inverse transformation kernels, respectively. The general discrete form of the forward and inverse two-dimensional transforms is denoted by 

\begin{align}
\label{eqn:generic_transform}
R(u,v) = 
\sum_{x=0}^{M-1}
\sum_{y=0}^{N-1}s(x,y)t_{f}(x,y,u,v)
&&
s(x,y) = 
\sum_{x=0}^{M-1}
\sum_{y=0}^{N-1}R(u,v)t_{i}(x,y,u,v)
\end{align}

\noindent where $M$ and $N$ are the dimensions of the image, $x$ and $y$ are coordinates of the image, $u = \{0,1,2,...,M-1\}$ and $v = \{0,1,2,...,N-1\}$ are called transform variables. The $t_{f}$ function is responsible for the forward domain change and the $t_{i}$ transfers the image back to the spatial domain. The domain switch allows different approaches to operations to be executed and present features of the image that could not be represented in the spatial domain. The convolution operation, for instance, turns itself into a simple matrix multiplication task on the Fourier Transform domain (which will be detailed in the following sections) and that solves the performance constraint.