Within the scope of this work, the main reason for introducing the concepts of optics is the imaging devices - the microscope, in particular. The structure of the apparatus will be described in \autoref{sec:light_microscopy}. Moreover, a substantially large amount of devices rely on optical lenses for imaging, with properties such as depth of field and depth of focus.

As stated by \citeonline{halliday2013fundamentals}, \emph{lenses} are objects consisting of a transparent material, with a certain refractive index, that are made of two spherical surfaces on which light propagates and suffers refraction. They are used in optical systems due to their capacity to create images as long as their refractive index is not equal to that of the medium. Yet, in agreement with \citeonline{halliday2013fundamentals}, some concepts related to lenses are important in our context and will be shown below. The following list depicts the principal elements from geometric optics that relates to lenses and its consequent imaging properties:

\begin{itemize}
    \item \emph{Radius of Curvature}: the distance between the center of the sphere and a refracting surface, named $\mathit{r_{1}}$ and $\mathit{r_{2}}$ on \autoref{fig:spherical_lens};
    
    \item \emph{Center of Curvature}: since the lenses are made by a union of two sections of a sphere-shaped object (which have a center), there are two Centers of Curvature for each lens, denoted by $\mathit{C_{1}}$ and $\mathit{C_{2}}$ on \autoref{fig:spherical_lens};
    
    \item \emph{Central Axis}: a line that represents the infinite number of radii of the sphere, which contain the center and the focal point;
    
    \item \emph{Focal Point}: also called \emph{focus}, a point within the central axis, where the image of the object is formed due to the convergence of the light rays from the object, and shown on \autoref{fig:spherical_lens} as $\mathit{F_{1}}$ and $\mathit{F_{2}}$;
    
    \item \emph{Focal Length}: presented as $\mathit{f}$ on \autoref{fig:spherical_lens}, it stands for the distance between the center of the sphere and the focal point;
    
    \item \emph{Object}: either a point or a surface on space that emits (or reflects) light and can be interpreted as a source;
    
    \item \emph{Image}: in this context, it is a representation of an object, formed by the action of lenses.
    
    \item \emph{Magnification}: a number that describes how larger or smaller the image will be in comparison to the object; mathematically, the ratio of the image distance to the object distance, both relative to the lens.
    
\end{itemize}

A single lens or a set of lenses (most of the optical systems are more complicated than a single lens) have the \emph{depth of field} and the \emph{depth of focus} properties. Although the terms appear to be similar, the former relates to objects and the latter, to images \cite{davidson2002optical}. For every system, there is a focal plane in which the formed images will be sharp. Depth of Field is the tolerance for the object focal plane that may produce sharp images, while Depth of Focus dictates the same tolerance for the image focal plane. In other words, depth of field is the zone in the real world that would yield an acceptably sharp image and depth of focus is the same idea for the imaging sensors or for plotting the image. \autoref{fig:spherical_lens} illustrates an arbitrary spherical lens.

\begin{figure}[htb]
	\centering
	\caption{\label{fig:spherical_lens} Arbitrary scheme of the optical properties of a spherical lens.}
	\begin{center}
	    \includegraphics[scale=0.4]{images/fig4.png}
	\end{center}
	\centering
    \fadaptada{halliday2013fundamentals}
\end{figure}

Some properties of the spherical lenses directly influence the image formation process and, consequently, the resulting image quality. The \emph{numerical aperture}, as reported by \citeonline{murphy2012fundamentals}, is a measurement in terms of angles that shows how much lenses can capture light, and is given by

\begin{equation}
    \label{eqn:numerical_aperture}
    NA = n \sin{\theta},
\end{equation}

\noindent where $\theta$ is the half angle of the cone of specimen light accepted by the objective lens and $n$ is the refractive index between the lens and the specimen. There are optical flaws in lenses that hinder a proper image formation. Those are named \emph{aberrations} \cite{lawlor2019introduction}, and the most relevant types in the scope of this project are the \emph{spherical aberrations}. According to \citeonline{murphy2012fundamentals}, the spherical aberrations occur when there is a difference in the focal point of incident parallel rays at central and peripheral locations of a spherical lens' surface, which yields a blurred image of either a point source of light or an extended object. It is possible to correct a spherical aberration by changing the shape of the refracting surface, i.e. changing the radius of curvature for the lenses in order to adjust the focal point to one particular distance \cite{smith1988optics}. The illustration in \autoref{fig:spherical_aberrations} represents the spherical aberration for a point source of light, where it is possible to see the difference between incident rays in the borders and in the inner regions of the lens. The resulting image is, in this case, a set of concentric circles around a point.

\begin{figure}[htb]
	\centering
	\caption{\label{fig:spherical_aberrations} Arbitrary example of the spherical aberration.}
	\begin{center}
	    \includegraphics[scale=0.3]{images/spherical_aberration.png}
	\end{center}
	\centering
    \fdireta{tipler2008physics}
\end{figure}