%%%%%%%% symbols to add
\simbolo{f(x,y)}{Original image as a function of spatial coordinates $x$ and $y$}

\simbolo{g(x,y)}{Observed image as a function of spatial coordinates $x$ and $y$}

\simbolo{J_{1}}{Bessel function of the first kind}

The human eye constructs images from incident light rays on the \emph{retina}, a very complex set of photoreceptors that converts light into electrical signals which are later interpreted by the brain. The result of this process may be modelled as continuous function of two variables $f(x,y)$ which comprises the \emph{illumination} and the \emph{reflectance} information, i.e. the amount of incident light and the amount of reflected light in the scene, respectively \cite{gonzalez2018digital}.

Digital image processing deals with, as the term suggests, \emph{digital images}, i.e. discrete representations of $f(x,y)$ generated by sensors that transform the illumination and reflectance information into electrical signals. Still according to \citeonline{gonzalez2018digital}, in order to achieve this representation, the signals undergo sampling (signal conversion from continuous to discrete) and quantization (mapping of real-valued intensities to discrete pixel values). For the sake of notation simplicity, $f(x,y)$ denotes the digital image and the term ``image'' also refer to it throughout this work.

It is clear that the image formation is influenced by several factors: sensor type, scene illumination conditions, and others. In fact, each imaging system such as a camera or a microscope adds its own constraints to the process, e.g. conventional transmitted light microscopy images are only achieved with non-opaque samples \cite{rudi2020contrast}. Although there are many types of microscopy, each with its own imaging procedure, this work limits its scope to bright-field microscopy. Therefore, this chapter summarizes the bright-field microscopy image formation processes and its implications on image quality. Furthermore, it describes blur properties concerning its origins either in image formation or other events. 

\section{Image Formation}
As mentioned in \autoref{chapter:fundamentals-of-optics-and-light-microscopy}, the bright-field microscopy images are formed by either transmitted or reflected light that passes through the sample and reaches the objective lens. The difference between transmitted light and reflected light microscopes is the illumination system; there is no difference in how both direct light rays leave the specimen \cite{leng2009materials}.

According to \citeonline{davidson2002optical}, the light which reaches the specimen is either undeviated, i.e. does not suffer any disturbances in its direction, or diffracted; the diffracted rays leave the sample with a phase difference of 180 degrees in comparison to the undeviated light and cause destructive interference in the eyepiece, which projects a magnified version of this pattern onto a sensor and consequently produces the image.

Furthermore, the diffraction patterns that are captured by objectives have a particular shape. As also stated by \citeonline{davidson2002optical}, the \emph{Airy disks}(also called \emph{Airy patterns}), named after Sir George Biddell Airy (1801 - 1892), are small circular diffraction disks projected by the objectives onto the image plane of the eyepiece diaphragm, which describe the focus profile the resulting image. The Airy disks, as described by \citeonline{fowles1989introduction}, follow the Fraunhoffer diffraction pattern, and may be mathematically modelled as an angular distribution of intensity of light diffracted by a circular aperture, given by

\begin{align}
\label{eqn:airy_function}
I(\theta) = I_{0} 
            \left[ 
            \frac{2 J_{1} (\rho)}{\rho}
            \right]^{2}
&&
\rho = \left( 
        \frac{2 \pi \sin{\theta}}{\lambda}
        \right) \frac{a}{2},
\end{align}

\noindent where $I_{0} = (C \pi R^{2})^{2}$ is the intensity for $\theta = 0$, $C$ is a constant, $R$ is the radius of the aperture, $\lambda$ is the wavelength of the light, $a$ is the diameter of the aperture and $J_{1}$ is the Bessel function of the first kind and first order \cite{mathews1970mathematical}. The Bessel function for the general case of $rth$ order is given by

\begin{align}
\label{eqn:1st_bessel}
J_{r}(x) = \sum_{n = 0}^{\infty}
            \frac{(-1)^{r}}
                 {r! \Gamma(m + r + 1)}
            \left(
                \frac{x}{2}
            \right)^{m + 2r}
&&
\Gamma(z) = \int_{0}^{\infty} e^{-u} u^{z-1}du.
\end{align}

The Airy disks are intrinsically related to the numerical aperture and the definition of \emph{resolution}. The resolution is the minimum distance between two points at which they can be visibly distinguished as two points; optically, it is defined as the minimum distance between two Airy disks that can be distinguished, which is limited by diffraction \cite{leng2009materials}. The resolution of an optical microscope, given by the Rayleigh equation, is described by

\begin{equation}
\label{eqn:resolution}
d = 1.22 \frac{\lambda}{2 NA},
\end{equation}

\noindent where $d$ is the space between two adjacent particles that may be distinguished from each other, $\lambda$ is the wavelength of the illumination and $NA$ is the numerical aperture of the objective \cite{davidson2002optical}. It is evident that objectives with higher numerical apertures and shorter wavelengths of visible light will yield better resolution. \autoref{fig:airy_disks} shows arbitrary examples of Airy patterns, as well as their possible configurations and their consequences to the image. In \autoref{fig:airy_disks}.(a), the usual shape of Airy patterns is shown, together with its two-dimensional representation as a function of the intensity by an interval. \autoref{fig:airy_disks}.(b) depicts an occurrence of Airy disk overlapping where both points would be properly resolved, i.e. below the Rayleigh limit, and \autoref{fig:airy_disks}.(c) represents the minimum distance in which both points would be distinguished. Finally, \autoref{fig:airy_disks}.(d) represents an unresolved pair of points.

\begin{figure}[htb]
	\centering
	\caption{\label{fig:airy_disks} Arbitrary example of an Airy disk (a), resolved Airy disks (b), Rayleigh limit of resolution (c) and unresolved Airy disks (d).} 
	\begin{center}
	    \includegraphics[scale=0.4]{images/airy_disks.png}
	\end{center}
	\centering
    \fadaptada{dunst2019imaging}
\end{figure}

As explained by \citeonline{goodman1996introduction}, an imaging system, particularly a set of microscope lenses, is said to be \emph{diffraction-limited} if the incident spherical light wave generated from a point-source object is transformed into another spherical wave which converges to an ideal image point, described by the original object point and affected by some sort of isotropic effect, such as magnification in a microscope.

The depth of field was described in \autoref{chapter:fundamentals-of-optics-and-light-microscopy} in terms of focal plane distances, but it may also be taken as the \emph{axial resolving power}, a measurement of resolution along the $z$ axis, determined by the numerical aperture and described by the Airy disk profile \cite{davidson2002optical}. Similarly to the Rayleigh's equation, the depth of field increases
with higher numerical apertures for the objective and shorter wavelengths of the incident light, and it represents a key property concerning the amount of blur in the resulting image.


\subsection{Point Spread Function and Image Formation Model}

\label{sec:point_spread_function_and_image_formation_model}

When light waves from a point source reach the lenses, they suffer diffraction and refraction, originating a new propagating set of rays that converge to a point in the center of the image plane in the shape of Airy disks; such shape is called the \sigla{PSF}{Point Spread Function}
of the imaging system (also called \emph{impulse response}), and it is intrinsically related to the imaging process \cite{wu2008microscope}. Particularly, the bright-field microscopy employs polychromatic nonpolarized incoherent light. Hence, it is possible to relate the Airy disks to the PSF, since those are intensity distributions for each point source of light emanating from the specimen. \autoref{fig:point_spread_function}.(a) presents a theoretical scheme of imaging for a point source of light, and \autoref{fig:point_spread_function}.(b) depicts the shape of a incoherent PSF:

\begin{figure}[htb]
	\centering
	\caption{\label{fig:point_spread_function} Point Spread Function generated by a focused diffraction-limited system with incoherent light.} 
	\begin{center}
	    \includegraphics[scale=0.235]{images/point_spread_function.jpeg}
	\end{center}
	\centering
    \fadaptada{castleman1996digital,wu2008microscope}
\end{figure}

In \autoref{fig:point_spread_function}.(a), the imaging system in in focus, which is given by

\begin{equation}
\label{eqn:lens_focus}
\frac{1}{d_{o}} + \frac{1}{d_{i}} = \frac{1}{f},
\end{equation}

\noindent where $f$ is the focal length of the lens, $d_{o}$ and $d_{i}$ are the distances from the point source plane to the lens and the distance from the image plane to the lens, respectively. The intensity of light in the point source is directly proportional to the intensity in the image, what characterizes a \emph{two-dimensional linear system} \cite{castleman1996digital}. Also according to \citeonline{castleman1996digital}, any motion of the point source on its plane moves the image is dictated by the law

\begin{align}
\label{eqn:isoplanatic_motion}
x_{i} = -\frac{d_{i}}{d_{o}}x_{o}
&&
y_{i} = -\frac{d_{i}}{d_{o}}y_{o},
\end{align}

\noindent where $(x_{o},y_{o})$ are the coordinates for the object location on its plane and $(x_{i},y_{i})$ are coordinates that locate the image on its plane. This implies that the shape of the image will not change according to the object's location, and this property yields \emph{shift invariance} to the system, which may be called \emph{isoplanatic}. These properties are observed in an ideal imaging system, not in real cases such as an optical microsope. Simple lenses are neither isoplanatic nor linear, however, there are approximations and mathematical tools that allow advanced microscopes to be assumed isoplanatic and linear.

The PSF, as related to the intensity distributions described by the Airy disks, are limited to the area of the aperture. This means that the amount of light that reaches the image plane is truncated by the circular aperture, what is also true for the PSF. The truncation is mathematically represented by the \emph{pupil function}, which is zero outside the boundaries of the aperture and unity otherwise, and might include also information about wave aberrations of the lens \cite{goodman1996introduction}. As denoted by \citeonline{wu2008microscope} with some notation adjustments, the PSF of an incoherent illuminated circular aperture imaging system is the Fourier Transform (explained further in \autoref{chapter:theoretical-background}) of the generalized pupil function, given by

\begin{equation}
\label{eqn:incoherent_psf}
h_{\lambda}(x,y,z) = \int_{-\infty}^{\infty}
                     \int_{-\infty}^{\infty}
         P(u,v)
         e^{j 2 \pi z
            \left(
                \frac{u^{2} + v^{2}}{2 \lambda L^{2}}        
            \right)
        }
        e^{j 2 \pi
            \left(
                \frac{xu + yv}{\lambda L}        
            \right)
        }
        du dv,
\end{equation}

\noindent where $h_{\lambda}(x,y,z)$ is point spread function for a light with wavelength $\lambda$, $P(x,y)$ is the pupil function, $z$ is the axial location the focal plane and $L = r / NA$ is the focal length, i.e. ratio between the radius of the circular aperture of the objective and the numerical aperture. The normalized Fourier Transform of the PSF is called \sigla{OTF}{Optical Transfer Function} \cite{castleman1996digital}.

The image is then formed as a set of impulse responses from each point in the object plane that were magnified by the imaging system. Linear systems possess a general expression, a convolution (which will be explained in \autoref{chapter:theoretical-background}) of the input with the system's impulse response, that describes the output \cite{brigham1988fast}. In this sense, the resulting image is a convolution of the PSF with the original image, defined as

\begin{equation}
\label{eqn:image_formation_convolution}
g(x,y) = \int_{-\infty}^{\infty}
         \int_{-\infty}^{\infty}
         h(x-u, y-v)f(x,y)du dv,
\end{equation}

\noindent where $f(x,y)$ is the original image, $g(x,y)$ is the observed image, $h(x,y)$ is the PSF of the imaging system and $u,v$ are shift parameters.

\subsection{Discrete Image Formation Model}

Digital images follow a discrete model for image formation due to the acquisition process: the spherical waves that leave the objectives reach the surface of \sigla{CCD}{Charge-coupled Devices}, sensors which proportionally convert light intensities to electrical signals digitized as pixels \cite{gonzalez2018digital}. The digital images are matrices of pixels that represent light intensities with different channel configurations, where the most common one is the \sigla{RGB}{Red, Green and Blue} image. Therefore, similarly to the image formation model shown in \autoref{sec:point_spread_function_and_image_formation_model}, the two-dimensional digital image formation is arbitrarily described as a discrete process

\begin{equation}
\label{eqn:discrete_image_formation}
g[x,y] = h[x,y] \ast f[x,y],
\end{equation}

\noindent where $\ast$ denotes the discrete convolution, $g$, $h$ and $f$ are respectively the observed image, the discrete PSF of the imaging system and the original image, and $x,y \in \mathbb{Z}$. The discrete PSF of an imaging system with incoherent illumination and a circular aperture is given by

\begin{align}
\label{eqn:discrete_psf}
h(r) = \left[
        2
        \frac{J_{1}[\pi (r / r_{0})]}{\pi (r / r_{0})}
       \right]^{2},
&&
r = \sqrt{x^{2} + y^{2}},
&&
r_{0} = \frac{\lambda d_{i}}{a},
\end{align}

\noindent where $h(r)$ is the radially symmetrical PSF, $r$ is the radial distance, $r_{0}$ is a scaling factor, $J_{1}$ is the Bessel function of first order and first kind, $\lambda$ is the wavelength of the illumination, $a$ is the diameter of the aperture and $d_{i}$ is the distance from the lens plane to the image plane. A scheme of the geometric setup of discrete image formation through the PSF is shown in \autoref{fig:discrete_psf_scheme}.

\begin{figure}[htb]
	\centering
	\caption{\label{fig:discrete_psf_scheme} Geometric scheme of a lens' circular aperture and arbitrary point spread function profile.} 
	\begin{center}
	    \includegraphics[scale=0.35]{images/discrete_psf_scheme.jpeg}
	\end{center}
	\centering
    \fadaptada{castleman1996digital}
\end{figure}

The discrete OTF is then the \sigla{DFT}{Discrete Fourier Transform} (explained in \autoref{chapter:theoretical-background}) of the PSF in \autoref{eqn:discrete_psf}. The OTF characterizes the intensities of light that emanate from the specimen in terms of frequencies. 

There is another property of image formation and acquisition that influences the image quality: the noise. Pursuant to \citeonline{wu2008microscope}, imaging is corrupted by intrinsic or extrinsic noise; the former is modelled by a Poisson distribution that influences each photon that reaches the sensor, and the latter is modelled by a Gaussian distribution that sums to the matrix of pixels. Details about noise are out of the scope of this work, since the degradation to be deeply explored is the defocus blur.

\section{Defocus Blur}
The blur effect is one type of degradation that consists of global or local information loss in the image. The defocus blur is caused by the incidence of light within an aperture with significant dimensions, where the source of light is not properly placed in accordance to the focal plane; it is related to the variables of the optical system such as depth of focus, aperture, depth of field, aberrations and so on \cite{joshi2014defocus}. According to \citeonline{smith2007modern}, every optical system exhibits blur properties, in higher or lower proportions, due to the depth of focus and its adjustment. Therefore, blurring is unavoidable to a certain extent, hence every imaging device possesses a PSF due to its optics. The PSF is also named \emph{blur kernel}.

Another useful way to mathematically describe the point spread function is through the Dirac Delta. It consists of a generalized function that represents an impulse, i.e. an infinitely high value within an infinitely small period of time \cite{bracewell2000fourier}. \autoref{fig:psf} shows an arbitrary example of a punctual source of light and its image, which suffers the spreading effect.

\begin{figure}[H]
	\centering
	\caption{\label{fig:psf} Magnified image of a light impulse (left) and its impulse response function, the PSF (right).}
	\begin{center}
	    \includegraphics[scale=0.4]{images/fig8.png}
	\end{center}
	\centering
    \fadaptada{gonzalez2008digital}
\end{figure}

Namely, it is a function $\delta(x)$ that is zero-valued for any $x \neq 0$ and is infinity-valued for $x = 0$. This property can be combined with any smooth function $f\colon \mathbb{R}^{n} \to \mathbb{R}^{n}$.
The continuous Dirac Delta may be written, as stated by \citeonline{weisstein2020delta}, as

\begin{align}
\label{eqn:dirac_delta_function}
\delta^{2}(x,y)= 
\begin{cases}
    \infty, & \text{if } x^{2} + y^{2} =0\\
    0, & \text{if } x^{2} + y^{2} \neq 0
\end{cases},
&&
\int_{-\infty}^{\infty}
\int_{-\infty}^{\infty}
\delta^{2}(x,y)dxdy = 1.
\end{align}

\noindent The discrete version of the Dirac Delta function consists of an infinite sum instead of the integral. This concept of impulse is the point source of light, concerning images. It provides the blur effect on images, as it promotes the diffusion of the acquired information.
